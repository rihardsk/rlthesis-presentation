% \documentclass{beamer}
\documentclass[xetex,mathserif]{beamer}
% \documentclass[xetex,mathserif,sans serif]{beamer}

%------------------------
% xelatex
\usepackage{fontspec}
\usepackage{xunicode}
\usepackage{xltxtra}

% languages
\usepackage{fixlatvian}
\usepackage{polyglossia}
\setdefaultlanguage{latvian}
%\setotherlanguages{english,russian}

\usepackage{graphicx} % Required for including images
\graphicspath{{figures/}} % Location of the graphics files
% \usepackage[font=tiny,labelfont=bf]{caption} % Required for specifying captions to tables and figures
% \setbeamerfont{caption}{size=\footnotesize}
\setbeamerfont{caption}{series=\normalfont,size=\fontsize{5}{7}} 
% \usepackage{caption}
% \captionsetup{font=scriptsize,labelfont=scriptsize}

% \renewcommand{\labelitemii}{$\star$}
%pseidokodam
\usepackage[boxed,linesnumbered]{algorithm2e}
\SetAlgorithmName{Algoritms}{}{Algoritmu saraksts}

\usepackage{float}

\title{Paredzošā stimulētā mācīšanās}
\author{Rihards Krišlauks \newline \small{darba vadītājs: Asoc.prof., Dr. dat. Jānis Zuters}}
\date{Rīga 2016}
\titlegraphic{\includegraphics[width=3cm]{lu-logo-full.png}}

\begin{document}
  % \maketitle
  \frame{\titlepage}
  \begin{frame}
    \frametitle{Ievads un mērķi}
    %Content goes here
    \begin{itemize} 
    \item Neirozinātnē -- paredzošā kodēšana vēsta, ka smadzenes cenšas
      paredzēt ienākošos sensoru signālus no iepriekš novērotā, minimizējot
      paredzēšanas kļūdu.
    \item Datorzinātnē -- stimulētā mācīšanās ļauj vispārīgi skatīties uz
      uzdevumiem, kur aģentam jāmācās, mijiedarbojoties ar vidi. 
    \item Actor-critic algoritmi šķiet labi piemēroti no paredzošās kodēšanas
      aizgūtu ideju realizēšanai.
    \end{itemize}
    \vspace{0.5cm}
    Mērķi
    \begin{itemize} 
    \item Izveidot stimulētās mācīšanās algoritmu, kas veido iekšēju vides
      modeli, lai paredzētu dažādus vides aspektus.
    \item Eksperimentāli noteikt, vai tas paātrina mācīšanos.
    \end{itemize}
  \end{frame}


  \begin{frame}
    \frametitle{Stimulētā mācīšanās}
    \framesubtitle{Īss ieskats}
    \begin{itemize}
      \item Aģents mijiedarbojas ar vidi. Veicot darbību $a_t$, nonāk vides
        stāvoklī $s_{t+1}$ un saņem atalgojumu $r_{t}$.
    \end{itemize}
    % \vspace{0.5cm}
    \begin{center}
      \includegraphics[height=3cm]{rl.pdf}
    \end{center}
    \begin{itemize}
      \item Mērķis ir izstrādāt stratēģiju $\pi$, lai maksimizēt saņemto atalgojumu
        \[
          E\left[\sum_{t=0}^{\infty}\gamma^t r_t\right],
        \]
        kur $r_t$ tiek iegūts aģentam vadoties pēc stratēģijas $\pi$ un $\gamma \in
    \left[0; 1\right)$ ir atlaides koeficients.
    \end{itemize}
    
    %More content goes here
  \end{frame}
  
  
  \begin{frame}
    \frametitle{Stimulētā mācīšanās}
    \framesubtitle{Risināšanas paņēmieni}
    \begin{itemize}
      \item Modeļa aproksimācijā mēģina tuvināti iegūt vides dinamikas parametrus.
      \item Vērtību aproksimācijā mēģina tuvināti iegūt stāvokļu vērtību funkciju.
      \item Stratēģijas aproksimācijā mēģina tiešā veidā nonākt pie optimālās
        stratēģijas.
    \end{itemize}
  \end{frame}


  \begin{frame}
    \frametitle{CACLA algoritms}
    \begin{itemize}
      \item Pieder pie stratēģijas aproksimācijas algoritmiem, konkrētāk, pie
        \textit{actor-critic} algoritmiem.
      \item Satur \textbf{kritiķa} komponenti, kas aproksimē stāvokļu vērtības
        funkciju, un \textbf{aktiera} komponenti, kas aproksimē optimālo stratēģiju.
      \item Abām komponentēm mijiedarbojoties, algoritms mācās.
        \begin{itemize}
          \item Stāvokļa vērtības pieaugums nosaka stratēģijas izmaiņas.
          \item Stratēģijas izmaiņas noved pie izmaiņām stāvokļu vērtībās.
        \end{itemize}
    \end{itemize}
  \end{frame}

  \begin{frame}
    \frametitle{CACLA algoritms}
    \framesubtitle{Pseidokods}
    \begin{itemize}
      \item Veic darbību atbilstoši esošajai stratēģijai
        \begin{equation*}
          a_t = \mathcal{N}\left(Ac_t(s_t), 0.05\right)
        \end{equation*}
      \item Novēro nākamo stāvokli $s_{t+1}$ un iegūto atalgojumu $r_{t}$
      \item Pielāgo stāvokļu vērtību funkciju
        \begin{itemize}
          \item Ja $s_{t+1}$ nav gala stāvoklis
            \begin{equation*}
              V_{t+1}(s_t) \xleftarrow{\alpha} r_t + \gamma V_t(s_{t + 1})
            \end{equation*}
          \item Ja $s_{t+1}$ ir gala stāvoklis
            \begin{equation*}
              V_{t+1}(s_t) \xleftarrow{\alpha} r_t
            \end{equation*}
        \end{itemize}
      \item Ja darbības rezultātā stāvokļa vērtība ir pieaugusi, t.i., $V_{t+1}(s_t)
        > V_t(s_t)$
        \begin{itemize}
          \item Pielāgo stratēģijas funkciju
            \begin{equation*}
              Ac_{t+1}(s_t) \xleftarrow{\beta} a_t
            \end{equation*}
        \end{itemize}
    \end{itemize}
  \end{frame}

  \begin{frame}
    \frametitle{CCACLA algoritmi}
    \framesubtitle{CCACLA pamatalgoritms}
    \begin{itemize}
      \item CCACLA algoritms apvieno aktiera un kritiķa komponenti vienā neironu
        tīklā.
      \item Vērtību funkcijas (kritiķa) un stratēģijas (aktiera) pielāgošanas
        nosacījumi saglabājas nemainīgi.
    \end{itemize}
    \begin{center}
      \hspace{2cm} \includegraphics[height=5cm]{ccacla-basic.pdf}
    \end{center}
  \end{frame}

  \begin{frame}
    \frametitle{CCACLA algoritmi}
    \framesubtitle{CCACLA ar stāvokļa paredzēšanu un CCACLA ar paredzēšanu nākotnē}
    \begin{itemize}
      \item CCACLA ar stāvokļa paredzēšanu cenšas uzminēt arī nākamo stāvokli.
      \item CCACLA ar paredzēšanu nākotnē paredz nākamo stāvokli un tā vērtību.
        % $V_t(s_{t-2}) \xleftarrow{\alpha} r_{t-1} + \gamma v_t$
    \end{itemize}
    \begin{center}
      \hspace{2cm} \includegraphics[height=6cm]{ccacla-state.pdf}
    \end{center}
  \end{frame}
  
  \begin{frame}
    \frametitle{Eksperimenti}
    \framesubtitle{Izmantotās vides}
    \begin{columns}[c]
    \begin{column}[c]{5cm}
      \begin{itemize}
        \item Cart-pole -- nepārtraukta stāvokļu telpa, nepārtraukta darbību telpa
        \item Acrobot -- nepārtraukta stāvokļu telpa, diskrēta darbību telpa
        \item Mountain car -- nepārtraukta stāvokļu telpa, diskrēta darbību telpa
      \end{itemize}

    \end{column}
    \begin{column}[c]{5cm}
      \includegraphics[height=2cm]{pole.png}

      \vspace{1cm}
      \includegraphics[height=2cm]{acrobot.png}

      \vspace{0.7cm}
      \includegraphics[height=2cm]{car.png}
    \end{column}
    \end{columns}
      \tiny{Attēli ņemti no http://library.rl-community.org/}
  \end{frame}

  \begin{frame}
    \frametitle{Eksperimenti}
    \framesubtitle{Aģentu konfigurācija}
      \begin{itemize}
        \item CACLA tika izmantoti 20 neironi gan stāvokļu vērtību funkcijas,
          gan stratēģijas funkcijas neironu tīkla slēptajā slānī.
        \item CCACLA algoritmos tika izmantoti 20 neironi apvienotā neironu
          tīkla slēptajā slānī.
        \item Visi algoritmi tika darbināti ar mācīšanās ātruma parametra
          vērtību $10^{-5}$ vai atsevišķos gadījumos ar $10^{-4}$.
      \end{itemize}
  \end{frame}

  \begin{frame}
    \frametitle{Eksperimenti}
    \framesubtitle{Cart-pole rezultāti}
    \begin{columns}[T]
    \begin{column}{.5\textwidth}
      % \begin{figure}
      \includegraphics[height=2.75cm]{{pole.cacla.all}.pdf}
      \captionof{figure}{CACLA iegūtais atalgojums}
      % \end{figure}

      % \vspace{0.5cm}
      % \begin{figure}
      \includegraphics[height=2.75cm]{{pole.ccacla-basic.all}.pdf}
      \captionof{figure}{CCACLA pamatalgoritma iegūtais atalgojums}
      % \end{figure}
    \end{column}
    \begin{column}{.5\textwidth}
      % \begin{figure}
      \includegraphics[height=2.75cm]{{pole.ccacla-state.all}.pdf}
      \captionof{figure}{CCACLA ar stāvokļa paredzēšanu iegūtais atalgojums}
      % \end{figure}

      % \begin{figure}
      \includegraphics[height=2.75cm]{{pole.ccacla-future.all}.pdf}
      \captionof{figure}{CCACLA ar paredzēšanu nākotnē iegūtais atalgojums}
      % \end{figure}
    \end{column}
    \end{columns}
  \end{frame}

  \begin{frame}
    \frametitle{Eksperimenti}
    \framesubtitle{Cart-pole apkopotie rezultāti}
      \begin{centering}
        \includegraphics[height=5.5cm]{{pole.all}.pdf}
        \captionof{figure}{Visu algoritmu vidējie rezultāti}
      \end{centering}
  \end{frame}

  \begin{frame}
    \frametitle{Eksperimenti}
    \framesubtitle{Acrobot rezultāti}
    \begin{columns}[T]
    \begin{column}{.5\textwidth}
      % \begin{figure}
      \includegraphics[height=2.75cm]{{acrobot.cacla.all}.pdf}
      \captionof{figure}{CACLA iegūtais atalgojums}
      % \end{figure}

      % % \vspace{0.5cm}
      % % \begin{figure}
      % \includegraphics[height=2.75cm]{{acrobot.ccacla-basic.all}.pdf}
      % \captionof{figure}{CCACLA pamatalgoritma iegūtais atalgojums}
      % % \end{figure}
    \end{column}
    \begin{column}{.5\textwidth}
      % \begin{figure}
      \includegraphics[height=2.75cm]{{acrobot.ccacla-state.all}.pdf}
      \captionof{figure}{CCACLA ar stāvokļa paredzēšanu iegūtais atalgojums}
      % \end{figure}

      % \begin{figure}
      \includegraphics[height=2.75cm]{{acrobot.ccacla-future.all}.pdf}
      \captionof{figure}{CCACLA ar paredzēšanu nākotnē iegūtais atalgojums}
      % \end{figure}
    \end{column}
    \end{columns}
  \end{frame}

  \begin{frame}
    \frametitle{Eksperimenti}
    \framesubtitle{Acrobot apkopotie rezultāti}
      \begin{centering}
        \includegraphics[height=5.5cm]{{acrobot.all}.pdf}
        \captionof{figure}{Visu algoritmu vidējie rezultāti}
      \end{centering}
  \end{frame}

  \begin{frame}
    \frametitle{Eksperimenti}
    \framesubtitle{Mountain car rezultāti}
    \begin{columns}[T]
    \begin{column}{.5\textwidth}
      % \begin{figure}
      \includegraphics[height=2.75cm]{{car.cacla.all}.pdf}
      \captionof{figure}{CACLA iegūtais atalgojums}
      % \end{figure}

      % \vspace{0.5cm}
      % \begin{figure}
      \includegraphics[height=2.75cm]{{car.ccacla-basic.all}.pdf}
      \captionof{figure}{CCACLA pamatalgoritma iegūtais atalgojums}
      % \end{figure}
    \end{column}
    \begin{column}{.5\textwidth}
      % \begin{figure}
      \includegraphics[height=2.75cm]{{car.ccacla-state.all}.pdf}
      \captionof{figure}{CCACLA ar stāvokļa paredzēšanu iegūtais atalgojums}
      % \end{figure}

      % \begin{figure}
      \includegraphics[height=2.75cm]{{car.ccacla-future.all}.pdf}
      \captionof{figure}{CCACLA ar paredzēšanu nākotnē iegūtais atalgojums}
      % \end{figure}
    \end{column}
    \end{columns}
  \end{frame}

  \begin{frame}
    \frametitle{Eksperimenti}
    \framesubtitle{Mountain car apkopotie rezultāti}
      \begin{centering}
        \includegraphics[height=5.5cm]{{car.all}.pdf}
        \captionof{figure}{Visu algoritmu vidējie rezultāti}
      \end{centering}
  \end{frame}
  
  \begin{frame}
    \frametitle{Secinājumi}
      \begin{itemize}
        \item Ideja par aģenta iekšēja vides modeļa veidošanu noveda pie:
          \begin{itemize}
            \item CACLA aktiera un kritiķa komponenšu apvienošanas;
            \item Idejas par nākamā stāvokļa un tā vērtības paredzēšanu.
          \end{itemize}
        \item Aktiera un kritiķa apvienošana vienotā neironu tīklā ļauj aktierim
          mācīties pat tad, kad tiek piegādāta informācija, kas sākotnēji bija
          derīga tikai kritiķim.
        \item Šāda netiešā veidā veikta aktiera parametru pielāgošana noved pie
          trokšņa parametru vērtību izmaiņās.
        \item Nākamā stāvokļa un tā vērtības paredzēšana ļauj ar šo troksni
          cīnīties.
      \end{itemize}
  \end{frame}
  
  \begin{frame}
    \begin{center}
    \huge{Paldies par uzmanību!}

    \vspace{0.5cm}
    \normalsize{Jautājumi?}
    \end{center}
  \end{frame}

% etc
\end{document}