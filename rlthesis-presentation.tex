% \documentclass{beamer}
\documentclass[xetex,mathserif]{beamer}
% \documentclass[xetex,mathserif,sans serif]{beamer}

%------------------------
% xelatex
\usepackage{fontspec}
\usepackage{xunicode}
\usepackage{xltxtra}

% languages
\usepackage{fixlatvian}
\usepackage{polyglossia}
\setdefaultlanguage{latvian}
%\setotherlanguages{english,russian}

\graphicspath{{figures/}} % Location of the graphics files

% \renewcommand{\labelitemii}{$\star$}

%pseidokodam
\usepackage[boxed,linesnumbered]{algorithm2e}
\SetAlgorithmName{Algoritms}{}{Algoritmu saraksts}

\usepackage{float}

\title{Paredzošā stimulētā mācīšanās}
\author{Rihards Krišlauks \newline \small{darba vadītājs: Asoc.prof., Dr. dat. Jānis Zuters}}
\date{Rīga 2016}
\titlegraphic{\includegraphics[width=3cm]{lu-logo-full.png}}

\begin{document}
  % \maketitle
  \frame{\titlepage}
  \begin{frame}
    \frametitle{Ievads un mērķi}
    %Content goes here
    \begin{itemize} 
    \item Neirozinātnē -- paredzošā kodēšana vēsta, ka smadzenes cenšas
      paredzēt ienākošos sensoru signālus no iepriekš novērotā, minimizējot
      paredzēšānas kļūdu.
    \item Datorzinātnē -- stimulētā mācīšanās ļauj vispārīgi skatīties uz
      uzdevumiem, kur aģentam jāmācās, mijiedarbojoties ar vidi. 
    \item Actor-critic algoritmi šķiet labi piemēroti no paredzošās kodēšanas
      aizgūtu ideju realizēšanai.
    \end{itemize}
    \vspace{0.5cm}
    Mērķi
    \begin{itemize} 
    \item Izveidot stimulētās mācīšanās algoritmu, kas veido iekšēju vides
      modeli, lai paredzētu dažādus vides aspektus.
    \item Eksperimentāli noteikt, vai tas paātrina mācīšanos.
    \end{itemize}
  \end{frame}


  \begin{frame}
    \frametitle{Stimulētā mācīšanās}
    \framesubtitle{Īss ieskats}
    \begin{itemize}
      \item Aģents mijiedarbojas ar vidi. Veicot darbību $a_t$, nonāk vides
        stāvoklī $s_{t+1}$ un saņem atalgojumu $r_{t}$.
    \end{itemize}
    % \vspace{0.5cm}
    \begin{center}
      \includegraphics[height=3cm]{rl.pdf}
    \end{center}
    \begin{itemize}
      \item Mērķis ir izstrādāt stratēģiju $\pi$, lai maksimizēt saņemto atalgojumu
    \end{itemize}
    \[
      E\left[\sum_{t=0}^{\infty}\gamma^t r_t\right],
    \]
    kur $r_t$ tiek iegūts aģentam vadoties pēc stratēģijas $\pi$ un $\gamma \in
    \left[0; 1\right)$ ir atlaides koeficients.
    
    %More content goes here
  \end{frame}
  
  
  \begin{frame}
    \frametitle{Stimulētā mācīšanās}
    \framesubtitle{Risināšanas paņēmieni}
    \begin{itemize}
      \item Modeļa aproksimācijā mēģina tuvināti iegūt vides dinamikas parametrus.
      \item Vērtību aproksimācijā mēģina tuvināti iegūt stāvokļu vērtību funkciju.
      \item Stratēģijas aproksimācijā mēģina tiešā veidā nonākt pie optimālās
        stratēģijas.
    \end{itemize}
  \end{frame}


  \begin{frame}
    \frametitle{CACLA algoritms}
    \begin{itemize}
      \item Pieder pie stratēģijas aproksimācijas algoritmiem, konktētāk, pie
        \textit{actor-critic} algoritmiem.
      \item Satur \textbf{kritiķa} komponenti, kas aproksimē stāvokļu vērtības
        funkciju, un \textbf{aktiera} komponenti, kas aproksimē optimālo stratēģiju.
      \item Abām komponentēm mijiedarbojoties, algoritms mācās.
        \begin{itemize}
          \item Stāvokļa vērtības pieaugums nosaka stratēģijas izmaiņas.
          \item Stratēģijas izmaiņas noved pie izmaiņām stāvokļu vērtībās.
        \end{itemize}
    \end{itemize}
  \end{frame}

  \begin{frame}
    \frametitle{CACLA algoritms}
    \framesubtitle{Pseidokods}
    \begin{itemize}
      \item Veic darbību atbilstoši esošajai stratēģijai
        \begin{equation*}
          a_t = \mathcal{N}\left(Ac_t(s_t), 0.05\right)
        \end{equation*}
      \item Novēro nākamo stāvokli $s_{t+1}$ un iegūto atlagojumu $r_{t}$
      \item Pielāgo stāvokļu vērtību funkciju
        \begin{itemize}
          \item Ja $s_{t+1}$ nav gala stāvoklis
            \begin{equation*}
              V_{t+1}(s_t) \xleftarrow{\alpha} r_t + \gamma V_t(s_{t + 1})
            \end{equation*}
          \item Ja $s_{t+1}$ ir gala stāvoklis
            \begin{equation*}
              V_{t+1}(s_t) \xleftarrow{\alpha} r_t
            \end{equation*}
        \end{itemize}
      \item Ja darbības rezultātā stāvokļa vērtība ir pieaugusi~--~$V_{t+1}(s_t)
        > V_t(s_t)$
        \begin{itemize}
          \item Pielāgo stratēģijas funkciju
            \begin{equation*}
              Ac_{t+1}(s_t) \xleftarrow{\beta_t(s_t)} a_t
            \end{equation*}
        \end{itemize}
    \end{itemize}
  \end{frame}

  \begin{frame}
    \frametitle{CCACLA algoritmi}
    \framesubtitle{CCACLA pamatalgoritms}
    \begin{itemize}
      \item CCACLA algoritms apvieno aktiera un kritiķa komponenti vienā neironu
        tīklā.
      \item Vērtību funkcijas (kritiķa) un stratēģijas (aktiera) pielāgošanas
        nosacījumi saglabājas nemainīgi.
    \end{itemize}
    \begin{center}
      \hspace{2cm} \includegraphics[height=5cm]{ccacla-basic.pdf}
    \end{center}
  \end{frame}

  \begin{frame}
    \frametitle{CCACLA algoritmi}
    \framesubtitle{CCACLA ar stāvokļa paredzēšanu un CCACLA ar paredzēšanu nākotnē}
    \begin{itemize}
      \item CCACLA ar stāvokļa paredzēšanu cenšas uzminēt arī nākamo stāvokli.
      \item CCACLA ar paredzēšanu nākotnē paredz nākamo stāvokli un tā vērtību.
        % $V_t(s_{t-2}) \xleftarrow{\alpha} r_{t-1} + \gamma v_t$
    \end{itemize}
    \begin{center}
      \hspace{2cm} \includegraphics[height=6cm]{ccacla-state.pdf}
    \end{center}
  \end{frame}
  
  \begin{frame}
    \frametitle{Eksperimenti}
    \framesubtitle{Izmantotās vides}
    \begin{columns}[C]
    \begin{column}[C]{5cm}
      \begin{itemize}
        \item Cart-pole -- nepārtraukta stāvokļu telpa, nepārtraukta darbību telpa
        \item Acrobot -- nepārtraukta stāvokļu telpa, diskrēta darbību telpa
        \item Mountain car -- nepārtraukta stāvokļu telpa, diskrēta darbību telpa
      \end{itemize}

    \end{column}
    \begin{column}[C]{5cm}
      \includegraphics[height=2cm]{pole.png}

      \vspace{1cm}
      \includegraphics[height=2cm]{acrobot.png}

      \vspace{0.7cm}
      \includegraphics[height=2cm]{car.png}
    \end{column}
    \end{columns}
      \tiny{Attēli ņemti no http://library.rl-community.org/}
  \end{frame}

  % \begin{frame}
  %   \frametitle{Eksperimenti}
  %   \framesubtitle{Cart-pole rezultāti}
  %   \begin{itemize}
  %     \item CCACLA ar stāvokļa paredzēšanu cenšas uzminēt arī nākamo stāvokli.
  %     \item CCACLA ar paredzēšanu nākotnē paredz nākamo stāvokli un tā vērtību.
  %       % $V_t(s_{t-2}) \xleftarrow{\alpha} r_{t-1} + \gamma v_t$
  %   \end{itemize}
  %   \begin{center}
  %     \hspace{2cm} \includegraphics[height=6cm]{ccacla-state.pdf}
  %   \end{center}
  % \end{frame}
% etc
\end{document}